% Options for packages loaded elsewhere
\PassOptionsToPackage{unicode}{hyperref}
\PassOptionsToPackage{hyphens}{url}
%
\documentclass[
  12pt,
  oneside]{article}
\usepackage{amsmath,amssymb}
\usepackage{setspace}
\usepackage{iftex}
\ifPDFTeX
  \usepackage[T1]{fontenc}
  \usepackage[utf8]{inputenc}
  \usepackage{textcomp} % provide euro and other symbols
\else % if luatex or xetex
  \usepackage{unicode-math} % this also loads fontspec
  \defaultfontfeatures{Scale=MatchLowercase}
  \defaultfontfeatures[\rmfamily]{Ligatures=TeX,Scale=1}
\fi
\usepackage{lmodern}
\ifPDFTeX\else
  % xetex/luatex font selection
    \setmainfont[]{Times New Roman}
\fi
% Use upquote if available, for straight quotes in verbatim environments
\IfFileExists{upquote.sty}{\usepackage{upquote}}{}
\IfFileExists{microtype.sty}{% use microtype if available
  \usepackage[]{microtype}
  \UseMicrotypeSet[protrusion]{basicmath} % disable protrusion for tt fonts
}{}
\makeatletter
\@ifundefined{KOMAClassName}{% if non-KOMA class
  \IfFileExists{parskip.sty}{%
    \usepackage{parskip}
  }{% else
    \setlength{\parindent}{0pt}
    \setlength{\parskip}{6pt plus 2pt minus 1pt}}
}{% if KOMA class
  \KOMAoptions{parskip=half}}
\makeatother
\usepackage{xcolor}
\usepackage[margin=1in]{geometry}
\usepackage{graphicx}
\makeatletter
\newsavebox\pandoc@box
\newcommand*\pandocbounded[1]{% scales image to fit in text height/width
  \sbox\pandoc@box{#1}%
  \Gscale@div\@tempa{\textheight}{\dimexpr\ht\pandoc@box+\dp\pandoc@box\relax}%
  \Gscale@div\@tempb{\linewidth}{\wd\pandoc@box}%
  \ifdim\@tempb\p@<\@tempa\p@\let\@tempa\@tempb\fi% select the smaller of both
  \ifdim\@tempa\p@<\p@\scalebox{\@tempa}{\usebox\pandoc@box}%
  \else\usebox{\pandoc@box}%
  \fi%
}
% Set default figure placement to htbp
\def\fps@figure{htbp}
\makeatother
\setlength{\emergencystretch}{3em} % prevent overfull lines
\providecommand{\tightlist}{%
  \setlength{\itemsep}{0pt}\setlength{\parskip}{0pt}}
\setcounter{secnumdepth}{5}
% definitions for citeproc citations
\NewDocumentCommand\citeproctext{}{}
\NewDocumentCommand\citeproc{mm}{%
  \begingroup\def\citeproctext{#2}\cite{#1}\endgroup}
\makeatletter
 % allow citations to break across lines
 \let\@cite@ofmt\@firstofone
 % avoid brackets around text for \cite:
 \def\@biblabel#1{}
 \def\@cite#1#2{{#1\if@tempswa , #2\fi}}
\makeatother
\newlength{\cslhangindent}
\setlength{\cslhangindent}{1.5em}
\newlength{\csllabelwidth}
\setlength{\csllabelwidth}{3em}
\newenvironment{CSLReferences}[2] % #1 hanging-indent, #2 entry-spacing
 {\begin{list}{}{%
  \setlength{\itemindent}{0pt}
  \setlength{\leftmargin}{0pt}
  \setlength{\parsep}{0pt}
  % turn on hanging indent if param 1 is 1
  \ifodd #1
   \setlength{\leftmargin}{\cslhangindent}
   \setlength{\itemindent}{-1\cslhangindent}
  \fi
  % set entry spacing
  \setlength{\itemsep}{#2\baselineskip}}}
 {\end{list}}
\usepackage{calc}
\newcommand{\CSLBlock}[1]{\hfill\break\parbox[t]{\linewidth}{\strut\ignorespaces#1\strut}}
\newcommand{\CSLLeftMargin}[1]{\parbox[t]{\csllabelwidth}{\strut#1\strut}}
\newcommand{\CSLRightInline}[1]{\parbox[t]{\linewidth - \csllabelwidth}{\strut#1\strut}}
\newcommand{\CSLIndent}[1]{\hspace{\cslhangindent}#1}

\usepackage{titlesec}
\titleformat{\section}{\normalfont\fontsize{14}{16}\bfseries}{\thesection}{1em}{}
\titleformat{\subsection}{\normalfont\fontsize{13}{15}\bfseries}{\thesubsection}{1em}{}
\titleformat{\subsubsection}{\normalfont\fontsize{12}{14}\bfseries}{\thesubsubsection}{1em}{}
\usepackage{bookmark}
\IfFileExists{xurl.sty}{\usepackage{xurl}}{} % add URL line breaks if available
\urlstyle{same}
\hypersetup{
  hidelinks,
  pdfcreator={LaTeX via pandoc}}

\author{}
\date{\vspace{-2.5em}}

\begin{document}


\begin{titlepage}
    \centering
    
    % University logo (replace with actual path)
    \IfFileExists{ictu-logo.png}{
        \includegraphics[width=0.5\textwidth]{ictu-logo.png}
    }{
        \vspace*{1cm}
        \textbf{\Large THE ICT UNIVERSITY}
    }
    
    \vspace{1cm}
    
    {\Large Faculty of Information and Communication Technology}
    
    \vspace{1.2cm}
    
    {A dissertation presented and submitted in partial fulfilment of the requirements\\
    for the degree of a Bachelor of Science in Computer Science}
    
    \vspace{0.5cm}

    {Titile}
    
    \textbf{Development of a Skin Lesion Detection and Classification System using Convolutional Neural Networks (CNNs)}
    
    \vspace{0.4cm}
    
    \textbf{By}
    
    \text{Ngane Emmanuel}
    
    \vspace{0.5cm}
    
    {Registration Number: ICTU20222972}
    
    \vspace{1cm}
    
    {Supervised by: Engr. Nkiamboh Tanwi}
    
    \vspace{2cm}
    
    \textbf{\today}
    
    \vfill
\end{titlepage}
\clearpage
\thispagestyle{empty}
\begin{center}
\textbf{\large DECLARATION}
\end{center}

\vspace{1cm}

I declare that the work entitled \textbf{``Development of a Skin Lesion Detection and Classification System using Convolutional Neural Networks (CNNs)''} is my own original work, conceived and presented in the partial fulfilment of the requirement for the degree of a Bachelor of Science in Computer Science at ICT University. This work has not been submitted for any degree or examination in any other university, and that all the sources I have used or quoted have been indicated and acknowledged as complete references.

\vspace{1.2cm}

\begin{tabular}{ll}
Signed \rule{4cm}{0.15mm} \hspace{3cm} & Date: \rule{4cm}{0.15mm} \\ [0.5cm]
Name: \rule{4cm}{0.15mm} & \\ [0.5cm]
Registration Number: \rule{4cm}{0.15mm} & \\
\end{tabular}
\vfill
\clearpage
\thispagestyle{empty}
\begin{center}
\textbf{\large CERTIFICATION}
\end{center}

\vspace{1cm}

This work entitled \textbf{``Development of a Skin Lesion Detection and Classification System using Convolutional Neural Networks (CNNs)''} has been submitted for examination with my approval as the Research Supervisor.

\vspace{2cm}

\begin{tabular}{ll}
Signed \rule{4cm}{0.15mm} \hspace{3cm} & Date: \rule{4cm}{0.15mm} \\ [0.5cm]
Name: \rule{4cm}{0.15mm} & \\
\end{tabular}
\clearpage

\thispagestyle{empty}
\begin{center}
\textbf{\large DEDICATION}
\end{center}

\vspace{2cm} % Adjust spacing as needed

I dedicate this work to my parents, Rev. Dr. Ntoko Samuel Eseh and Mme Ntoko Grace Melioge, and to my loving sister, Ntoko Racheal Edenge, for their unwavering support, encouragement, and sacrifices throughout my academic journey.

\vfill % Pushes the text to the vertical center
\clearpage

\thispagestyle{empty}
\begin{center}
\textbf{\large Acknowledgments}
\end{center}

\vspace{2cm} % Adjust spacing as needed

I would like to express my sincere appreciation to my project supervisor, Engr. Nkiambo Tanwi, whose guidance, feedback, and encouragement were invaluable throughout the course of this project. His expertise and support helped shape both the direction and quality of this research.

I am also grateful to the faculty and staff of the Department of Computer Science, ICT University, for providing the academic foundation and resources necessary for the completion of this work. Their commitment to academic excellence has been instrumental in my development.

Special thanks go to my classmates and friends for their collaboration, feedback, and moral support during the challenging phases of this project. Their insights and motivation helped me stay focused and persistent.

Lastly, I appreciate my family for their continuous encouragement and understanding, which enabled me to dedicate the necessary time and energy to this project.

This project has been a significant learning experience, and I am thankful to all who contributed to its successful completion.

\vfill % Pushes the text to the vertical center
\clearpage

\thispagestyle{empty}
\begin{center}
\textbf{\large FACULTY APPROVAL}
\end{center}

\vspace{1cm}

This dissertation has been duly reviewed by the Department and the Faculty and is ready for examination with our approval.

\vspace{2cm}

\begin{flushright}
\textbf{Approved by}

\vspace{1.5cm}

\begin{tabular}{@{}r@{}}
Signature \hspace{3cm} Date \\ [0.5cm]
\rule{6cm}{0.15mm} \\ [0.2cm]
Engr. Nkiamboh Tanwi \\ [0.3cm]
Supervisor \\[1cm]

Signature \hspace{3cm} Date \\ [0.5cm]
\rule{6cm}{0.15mm} \\ [0.2cm]
Dr. Abdallah Ziraba \\ [0.3cm]
Head of Department \\[1cm]

Signature \hspace{3cm} Date \\ [0.5cm]
\rule{6cm}{0.15mm} \\ [0.2cm]
Dr. Luc Einstein Ngend \\ [0.3cm]
Dean \\
\end{tabular}
\end{flushright}

\vfill
\clearpage

% Abstract Page
\thispagestyle{empty}
\begin{center}
\textbf{\Large ABSTRACT}
\end{center}

\vspace{1cm}

Skin lesions encompass a wide spectrum of conditions, ranging from benign disorders such as eczema, fungal infections, and psoriasis to malignant forms like melanoma and squamous cell carcinoma. Globally, these conditions impose significant public health and economic burdens, with their impact being particularly acute in low-resource regions where specialized dermatological expertise is scarce. In Cameroon and much of sub-Saharan Africa, access to timely and accurate dermatological diagnosis is hindered by limited specialist availability, under-resourced health facilities, and geographic barriers to care. 

This study presents the development of a multi-class skin lesion detection and classification system using Convolutional Neural Networks (CNNs), designed to enhance diagnostic accessibility through automated image analysis. The system was trained on a curated, high-quality dataset compiled from multiple open-access sources, including Kaggle dermatology repositories and the DermNetNZ medical image database. Careful preprocessing and augmentation techniques were employed to normalize image quality, address class imbalance, and improve model robustness across diverse lesion presentations and skin tones.

A fine-tuned ResNet-18 architecture, leveraging transfer learning from the ImageNet dataset, was implemented to classify lesions into multiple diagnostic categories, covering both malignant and non-malignant conditions. The model was trained using supervised learning with categorical cross-entropy loss, and evaluated using metrics including accuracy, precision, recall, and F1-score. The final system achieved a validation accuracy of 75\% and demonstrated consistent performance on the test set, underscoring its potential as a supportive diagnostic aid.

The implementation includes a command-line interface (CLI) for research use and a prototype mobile application framework aimed at offline deployment in rural and underserved settings. By combining computational efficiency with diagnostic breadth, the system offers a cost-effective and scalable approach to skin lesion classification in contexts where traditional healthcare access is limited. 

This project contributes to the growing field of AI-assisted dermatology by emphasizing inclusivity across skin tones and lesion types, while maintaining technical feasibility for low-resource environments. Future improvements will focus on expanding the dataset, integrating explainable AI features, incorporating multi-modal patient data, and validating the system in real-world clinical settings.

\vspace{1cm}
\noindent\textbf{Keywords:} Skin lesion detection, convolutional neural networks, ResNet-18, transfer learning, dataset diversity, mobile health, Cameroon, medical imaging.

\vfill
\clearpage

{
\setcounter{tocdepth}{3}
\tableofcontents
}
\setstretch{1.5}
\newpage

\listoffigures
\newpage

\newpage

\section{Chapter 1: Introduction}\label{chapter-1-introduction}

\subsection{Introduction}\label{introduction}

Skin lesions encompass a broad spectrum of abnormalities in the skin,
ranging from benign conditions such as moles and warts to malignant
manifestations like melanoma and squamous cell carcinoma. Globally,
these dermatological disorders present a significant public health
challenge, with skin cancers alone accounting for millions of new cases
annually and exerting substantial socioeconomic and healthcare burdens
(\citeproc{ref-ogudo2023optimal}{Ogudo et al., 2023}),
(\citeproc{ref-kassem2021machine}{Kassem et al., 2021}). The prevalence
of skin lesions has been amplified by factors such as increased
ultraviolet (UV) exposure due to climate change, demographic
transitions, and lifestyle changes, contributing to both malignant and
non-malignant skin disorders (\citeproc{ref-zafar2023skin}{Zafar et al.,
2023}).

In high-income countries, early diagnosis is often facilitated by access
to trained dermatologists, advanced imaging technologies, and
well-established referral systems. However, in low- and middle-income
countries (LMICs), particularly in sub-Saharan Africa, dermatology
services remain sparse, with a severe shortage of qualified
dermatologists and specialized diagnostic tools
(\citeproc{ref-hay2006skin}{Hay et al., 2006}). This disparity often
leads to delayed diagnoses, inappropriate treatments, and preventable
morbidity and mortality from conditions that are otherwise treatable
when detected early.

Cameroon, like many African nations, faces a dual challenge: the burden
of infectious skin diseases such as fungal infections, scabies, and
bacterial dermatitis coexists with a growing incidence of
non-communicable skin disorders, including actinic keratosis, psoriasis,
and malignant lesions (\citeproc{ref-van2005common}{Hees \& Naafs,
2005}). The limited integration of dermatological screening into primary
healthcare, coupled with the absence of large-scale public awareness
campaigns, exacerbates the problem.

Recent advancements in artificial intelligence (AI) and machine learning
(ML), particularly deep learning-based image analysis, have demonstrated
remarkable potential in automating the detection and classification of
skin lesions with performance levels approaching or surpassing that of
experienced dermatologists (\citeproc{ref-khan2021multi}{Khan et al.,
2021}), (\citeproc{ref-adegun2020fcn}{Adegun \& Viriri, 2020}). Such
systems leverage large annotated datasets to learn discriminative
features directly from dermoscopic or clinical images, enabling fast,
scalable, and cost-effective screening.

This research seeks to harness these technological advancements to
develop a skin lesion detection and classification system capable of
accurately identifying multiple common skin lesion types. The proposed
system integrates a convolutional neural network (CNN) trained on
diverse, high-quality datasets sourced from publicly available
repositories such as Kaggle and curated medical image databases like
DermNetNZ. In addition to the AI model, the system will include a
graphical user interface (GUI) for clinical use, a command-line
interface (CLI) for research purposes, and a mobile application to
extend accessibility to rural and underserved communities. This
comprehensive approach aims to bridge the diagnostic gap, improve early
detection rates, and contribute to the broader objective of reducing the
dermatological disease burden in Cameroon and similar contexts.

\subsection{Background to the Problem}\label{background-to-the-problem}

Dermatological diseases represent one of the most common categories of
health complaints worldwide, with conditions ranging from self-limiting
skin irritations to life-threatening malignancies. The World Health
Organization (WHO) has repeatedly emphasized that skin diseases are
among the top ten causes of non-fatal disease burden globally, and in
certain regions of sub-Saharan Africa, they are among the most frequent
reasons for outpatient consultations (\citeproc{ref-hay2006skin}{Hay et
al., 2006}), (\citeproc{ref-van2005common}{Hees \& Naafs, 2005}). While
conditions like melanoma and basal cell carcinoma dominate skin cancer
statistics in developed nations, the disease landscape in Africa and
particularly Cameroon is more heterogeneous, encompassing a mix of
infectious, inflammatory, and neoplastic disorders.

Traditionally, diagnosis of skin lesions has relied heavily on visual
inspection by a dermatologist, often supplemented by dermoscopic
evaluation or histopathological examination for suspicious cases
(\citeproc{ref-yap2018multimodal}{Yap et al., 2018}). However, this
process is inherently subjective and highly dependent on the clinician's
expertise and experience. In many LMICs, including Cameroon, the ratio
of dermatologists to the population is critically low, often less than
one dermatologist per million inhabitants, leading to substantial
diagnostic delays. Moreover, general practitioners and nurses, who often
serve as first-line healthcare providers, may lack specialized training
in dermatology, further compounding the risk of misdiagnosis or missed
diagnoses.

The advent of AI in medical imaging, particularly the application of
convolutional neural networks (CNNs) to dermoscopic and clinical images,
has opened new possibilities for addressing these challenges
(\citeproc{ref-harangi2018skin}{Harangi, 2018}),
(\citeproc{ref-mahbod2019skin}{Mahbod et al., 2019}). CNNs can
automatically extract hierarchical features from raw image data,
allowing them to differentiate between lesion types with minimal human
intervention. Studies have demonstrated that AI-based diagnostic tools
can achieve diagnostic accuracy comparable to or exceeding that of
experienced dermatologists in controlled settings
(\citeproc{ref-albahar2019skin}{Albahar, 2019}),
(\citeproc{ref-jinnai2020development}{Jinnai et al., 2020}).

Despite these advancements, most existing AI models for skin lesion
classification have been trained and validated on datasets that
predominantly contain images from lighter-skinned populations in Europe,
North America, and Australia (\citeproc{ref-gouda2022detection}{Gouda et
al., 2022}). This raises concerns about model generalizability to
populations with darker skin tones, such as those in sub-Saharan Africa.
Additionally, existing research has often focused on detecting malignant
lesions, particularly melanoma, with limited emphasis on the broader
spectrum of common lesions prevalent in African contexts, such as
eczema, tinea infections, or Kaposi's sarcoma.

This project responds to these gaps by developing a multi-class lesion
classification system specifically curated to include a diverse set of
common skin conditions affecting African populations. The dataset
compilation process involves selecting high-quality images from multiple
Kaggle repositories and supplementing them with clinically verified
images from DermNetNZ, ensuring representation across various lesion
categories and skin tones. Through this approach, the proposed system
aims to offer a clinically relevant, context-sensitive diagnostic
support tool that addresses both the technical challenges of AI
implementation and the pressing healthcare needs of the target
population.

\subsection{Problem Statement}\label{problem-statement}

The early and accurate diagnosis of skin lesions remains a critical
public health challenge in Cameroon and across much of sub-Saharan
Africa. While skin lesions may appear trivial in their initial stages,
certain forms such as malignant melanomas, squamous cell carcinomas, and
basal cell carcinomas, can rapidly progress to life-threatening stages
if left untreated. In parallel, other non-malignant but common
conditions, such as fungal infections, eczema, and psoriasis, can cause
significant discomfort, social stigma, and economic loss due to
chronicity and recurrence.

In many urban centers of high-income countries, advanced diagnostic
services and dermatology specialists are accessible, enabling early
detection and management. However, in Cameroon, the number of
dermatologists is grossly inadequate relative to the population size,
with some regions having no specialist at all
(\citeproc{ref-hay2006skin}{Hay et al., 2006}),
(\citeproc{ref-van2005common}{Hees \& Naafs, 2005}). This shortage
forces patients to rely on general practitioners or traditional healers,
which often leads to delayed diagnoses, mismanagement, or complete
neglect of skin-related health issues.

Furthermore, conventional diagnostic workflows are limited by two key
barriers:\\
1. \textbf{Geographic accessibility} -- Many patients in rural areas
must travel long distances to reach healthcare facilities, resulting in
missed opportunities for early diagnosis.\\
2. \textbf{Resource limitations} -- Even in urban hospitals, dermoscopic
imaging equipment and histopathological services are often scarce or
prohibitively expensive.

Recent developments in artificial intelligence, specifically
convolutional neural networks (CNNs), have shown promise in automating
skin lesion detection and classification with high accuracy
(\citeproc{ref-khan2021multi}{Khan et al., 2021}),
(\citeproc{ref-adegun2020fcn}{Adegun \& Viriri, 2020}). Nonetheless,
most existing models are trained on datasets from lighter-skinned
populations and focus primarily on malignant lesions, limiting their
applicability in African contexts where lesion diversity and skin tone
variations differ significantly (\citeproc{ref-gouda2022detection}{Gouda
et al., 2022}).

The central problem, therefore, is the absence of an accessible,
accurate, and context-specific diagnostic tool for multiple types of
skin lesions prevalent in Cameroon. There is a need for an integrated
system that combines AI-powered lesion detection with user-friendly
interfaces, ranging from clinical desktop applications to mobile apps
that can bridge the diagnostic gap between rural patients, urban
hospitals, and research institutions.

\subsection{Objectives of the Study}\label{objectives-of-the-study}

The overarching aim of this study is to develop a comprehensive skin
lesion detection and classification system that leverages deep learning
techniques to address the diagnostic challenges faced in Cameroon and
similar low-resource settings.

\subsubsection{General Objective}\label{general-objective}

To design and implement an AI-driven system capable of accurately
detecting and classifying multiple types of common skin lesions, with
deployment across desktop, command-line, and mobile platforms to improve
diagnostic accessibility.

\subsubsection{Specific Objectives}\label{specific-objectives}

\begin{enumerate}
    \item To compile and curate a diverse, high-quality dataset of common skin lesion images from multiple open-access sources, including Kaggle repositories and the DermNetNZ medical image database.
    \item To preprocess and augment image data to enhance model robustness..
    \item To train a convolutional neural network (CNN) model optimized for multi-class lesion classification, ensuring high performance on both malignant and non-malignant categories.
    \item To develop a command-line interface (CLI) for research and model testing purposes.
    \item To design and implement a mobile application capable of performing on-device lesion classification for remote and rural healthcare access.
    \item To evaluate the system's performance against standard diagnostic accuracy metrics, including sensitivity, specificity, and overall classification accuracy.
    \item To assess the feasibility of integrating the system into existing telemedicine frameworks in Cameroon.
\end{enumerate}

\subsection{Scope of the Study}\label{scope-of-the-study}

This study focuses on the development of a skin lesion detection and
classification system tailored for the Cameroonian healthcare context,
while incorporating global best practices in AI-assisted dermatological
diagnosis. The scope of the research spans four key dimensions: the
dataset, the technical solution, the deployment platforms, and the
evaluation process.

From a \textbf{data perspective}, the system will be trained and
validated using a carefully curated dataset compiled from multiple
publicly available repositories on Kaggle, supplemented with
high-quality dermatological images sourced from DermNetNZ. These sources
offer a broad spectrum of lesion categories, ranging from malignant
cancers such as melanoma and squamous cell carcinoma to non-malignant
conditions like eczema, psoriasis, and fungal infections. Selection
criteria will prioritize image clarity, correct annotation, diversity of
skin tones, and representation of lesion variations to ensure the
model's applicability across a wide demographic.

From a \textbf{technical perspective}, the system will employ a
convolutional neural network (CNN) architecture fine-tuned for
multi-class classification. Advanced preprocessing techniques, such as
data augmentation and normalization, will be applied to enhance the
model's generalizability across different imaging conditions. While the
primary focus is on image-based diagnosis, the system is not intended to
replace histopathological confirmation, which remains the gold standard
for definitive lesion classification.

From a \textbf{deployment perspective}, the project encompasses three
interfaces:

\begin{enumerate}
\def\labelenumi{\arabic{enumi}.}
\tightlist
\item
  A \textbf{Command-Line Interface (CLI)}, primarily intended for
  researchers and developers for testing and evaluating the AI model.\\
\item
  A \textbf{Mobile Application}, optimized for both offline and online
  usage, to enable healthcare workers and patients in rural areas to
  perform preliminary lesion assessments without constant internet
  connectivity.
\end{enumerate}

From an \textbf{evaluation perspective}, the model will be tested using
standard performance metrics, including accuracy, sensitivity,
specificity, and confusion matrix analysis. The evaluation will
emphasize the system's ability to handle variations in lesion
presentation due to differences in skin pigmentation, image quality, and
environmental lighting.

The project does not aim to cover the entire spectrum of dermatological
conditions, rare or highly complex lesions requiring specialized
diagnostic equipment fall outside the intended operational scope.
Instead, the focus is on delivering a reliable, accessible, and scalable
solution for the most common and clinically significant skin lesions
encountered in Cameroon and similar resource-limited environments.

\subsection{Significance of the Study}\label{significance-of-the-study}

Skin diseases are among the most common health concerns globally, with
billions of people affected each year, spanning all socioeconomic groups
(\citeproc{ref-hay2006skin}{Hay et al., 2006}). In Cameroon and much of
sub-Saharan Africa, the impact of skin lesions is amplified by a
combination of high prevalence, limited specialist care, and widespread
misinformation about skin health. Misdiagnosed or untreated lesions,
whether malignant or non-malignant, can lead to severe health
complications, disfigurement, psychological distress, and in the case of
cancers, increased mortality rates.

The significance of this study lies in its potential to bridge the
diagnostic gap through a context-specific, AI-driven approach. By
leveraging deep learning algorithms and multi-platform deployment, the
system offers a means of providing timely and accurate lesion assessment
to populations that currently lack access to dermatological expertise.
For healthcare providers, this could translate to earlier interventions,
better patient outcomes, and a reduction in the burden on tertiary
healthcare facilities.

From a \textbf{public health perspective}, the project addresses a
critical need in preventive care. Early detection and classification can
significantly reduce treatment costs and improve survival rates in
malignant cases, while minimizing chronic complications in non-malignant
cases. Moreover, the system's mobile deployment makes it a practical
tool for community health workers conducting outreach in rural and
peri-urban areas.

From a \textbf{technological perspective}, the research contributes to
the growing field of AI in medical imaging by demonstrating how existing
machine learning methods can be adapted to underrepresented populations.
Many existing AI models for skin lesion classification are trained
predominantly on lighter-skinned datasets, making them less effective
for darker skin tones (\citeproc{ref-gouda2022detection}{Gouda et al.,
2022}). By incorporating diverse skin tone representation from datasets
like DermNetNZ and Kaggle repositories, this study advances the
inclusivity and fairness of medical AI systems.

From an \textbf{academic perspective}, the project provides a valuable
reference for future studies in both computer vision and health
informatics within the African context. The integration of a
research-oriented CLI, a mobile health application, and a clinical GUI
sets a precedent for multi-platform medical AI systems designed for
low-resource settings.

Ultimately, the study seeks to contribute toward Sustainable Development
Goal 3 (Good Health and Well-being) by enabling accessible, affordable,
and high-quality dermatological diagnostics in regions where such
services are currently limited or absent.

\subsubsection{1.7 Limitations of the
Study}\label{limitations-of-the-study}

While this study seeks to develop a comprehensive, AI-powered skin
lesion detection and classification system, several constraints
inevitably shape the scope and potential impact of the research. First,
the dataset, although carefully curated from high-quality open-access
sources such as Kaggle repositories and the DermNetNZ medical image
database, may not capture the full diversity of skin lesion
presentations across all ethnic groups and age categories. This
limitation is particularly relevant for sub-Saharan African populations,
where variations in skin pigmentation can influence lesion visibility
and morphology, potentially impacting model generalization
(\citeproc{ref-hay2006skin}{Hay et al., 2006}),
(\citeproc{ref-van2005common}{Hees \& Naafs, 2005}).

Second, although image preprocessing and augmentation techniques are
employed to mitigate overfitting and improve robustness, the absence of
histopathological confirmation for all dataset images introduces an
inherent diagnostic uncertainty. Clinical image-based diagnosis, while
valuable, cannot entirely substitute for biopsy-confirmed ground truth,
especially in differentiating visually similar lesion types
(\citeproc{ref-zafar2023skin}{Zafar et al., 2023}).

Third, computational resource limitations constrain the complexity and
scale of model experimentation. While transfer learning with established
CNN architectures, such as ResNet, offers strong baseline performance,
exploring more computationally intensive models or ensemble methods may
be restricted due to hardware constraints. This also impacts the breadth
of hyperparameter optimization that can be conducted within the project
timeline.

Fourth, the current implementation focuses primarily on classification
accuracy and does not yet incorporate a complete clinical decision
support framework, such as integration with electronic health records
(EHRs) or automated referral systems. Similarly, the system's deployment
is confined to a mobile application and a research-oriented command-line
interface, without the immediate inclusion of a graphical interface for
clinical integration, an enhancement reserved for future development
phases.

Finally, while the study includes performance evaluation using standard
diagnostic metrics (e.g., sensitivity, specificity, overall accuracy),
real-world clinical validation with dermatologists or in live
telemedicine settings is beyond the scope of the present work. This
limits immediate translation into routine healthcare workflows but
provides a foundation for subsequent validation studies.

\subsubsection{1.8 Organization of the
Study}\label{organization-of-the-study}

The remainder of this thesis is organized into five chapters, each
systematically addressing a core component of the research.

\begin{itemize}
\tightlist
\item
  \textbf{Chapter 1 -- Introduction}: Provides an overview of the
  research problem, contextual background, study objectives, scope,
  significance, and limitations, setting the stage for the
  investigation.
\item
  \textbf{Chapter 2 -- Literature Review}: Presents a comprehensive
  examination of existing studies on skin lesion detection and
  classification, machine learning and deep learning techniques applied
  to medical imaging, and relevant mobile health (mHealth) application
  frameworks. This chapter critically evaluates the strengths and
  weaknesses of prior approaches, identifying gaps that this study seeks
  to address.
\item
  \textbf{Chapter 3 -- Methodology}: Describes the research design,
  dataset collection and curation process, preprocessing and
  augmentation strategies, CNN architecture selection, training
  procedures, and performance evaluation metrics. It also outlines the
  mobile application and CLI implementation details.
\item
  \textbf{Chapter 4 -- Results and Discussion}: Reports and analyzes
  experimental results, including classification performance, error
  analysis, and comparative evaluation against existing methods. The
  discussion interprets these findings in the context of the study
  objectives and broader literature.
\item
  \textbf{Chapter 5 -- Conclusion and Future Work}: Summarizes key
  findings, reiterates the contributions of the study, and outlines
  limitations alongside recommended directions for future research,
  including potential clinical integration and large-scale deployment
  strategies.
\end{itemize}

This structured organization ensures a logical progression from problem
identification through methodological implementation to empirical
validation and future considerations, facilitating clarity and coherence
for both technical and non-technical readers.

\newpage

\section{Chapter 2: Literature
Review}\label{chapter-2-literature-review}

\subsection{Introduction}\label{introduction-1}

The human skin, as the largest organ of the body, serves as the first
line of defense against environmental insults, pathogens, and physical
trauma. It performs essential physiological roles including
thermoregulation, sensory perception, and immunological protection.
However, the skin is susceptible to a wide range of pathological
conditions, collectively referred to as skin lesions, which can
significantly impair an individual's quality of life and, in severe
cases, become life-threatening (\citeproc{ref-hay2006skin}{Hay et al.,
2006}). These lesions may arise from infectious agents, autoimmune
disorders, genetic anomalies, or prolonged environmental exposure, with
their clinical presentation varying from benign and self-limiting forms
to aggressive malignancies (\citeproc{ref-van2005common}{Hees \& Naafs,
2005}).

Globally, skin diseases constitute a substantial public health
challenge, ranking among the top causes of non-fatal disease burden.
According to the Global Burden of Disease Study, conditions such as
eczema, acne, and fungal infections collectively affect billions of
individuals worldwide, while skin cancers contribute significantly to
morbidity and mortality in certain regions
(\citeproc{ref-zafar2023skin}{Zafar et al., 2023}). In sub-Saharan
Africa, and specifically in Cameroon, the prevalence of skin lesions is
amplified by climatic factors such as high ultraviolet (UV) index,
tropical humidity, and widespread infectious disease exposure, combined
with limited access to dermatological specialists and diagnostic
facilities (\citeproc{ref-hay2006skin}{Hay et al., 2006}). Rural
communities often face the most acute challenges, where early detection
and treatment are hindered by geographical, infrastructural, and
socioeconomic constraints.

Recent advances in Artificial Intelligence (AI), particularly in
computer vision, have opened promising avenues for automated skin lesion
analysis. Machine learning and deep learning techniques, particularly
Convolutional Neural Networks (CNNs), have demonstrated superior
performance in differentiating between lesion types, sometimes rivalling
trained dermatologists in diagnostic accuracy
(\citeproc{ref-kassem2021machine}{Kassem et al., 2021}),
(\citeproc{ref-ogudo2023optimal}{Ogudo et al., 2023}). Leveraging such
tools in low-resource settings could significantly enhance early
diagnosis, improve treatment outcomes, and reduce the long-term health
and economic burdens associated with skin diseases.

This chapter reviews the existing literature on skin lesion types, their
epidemiology, and the role of AI in dermatological diagnostics, with
particular emphasis on the African context. It also outlines the
technical underpinnings of CNN-based approaches, dataset challenges, and
existing AI-enabled systems for skin lesion detection.

\subsection{Overview of Skin Lesions and Skin
Cancer}\label{overview-of-skin-lesions-and-skin-cancer}

Skin lesions represent a broad spectrum of structural or functional
abnormalities affecting the skin's epidermal, dermal, or subcutaneous
layers. They are typically classified according to their clinical
appearance, etiology, and pathological significance, ranging from
benign, non-cancerous conditions to malignant cancers that pose
significant health risks (\citeproc{ref-zafar2023skin}{Zafar et al.,
2023}). Understanding the characteristics, prevalence, and diagnostic
challenges associated with each category is crucial for designing
AI-assisted diagnostic systems capable of reliable, multi-class
classification across diverse lesion types.

\subsubsection{Non-Cancerous Lesions}\label{non-cancerous-lesions}

Non-cancerous lesions comprise the majority of dermatological cases
encountered in both primary and specialized care. These include
inflammatory conditions such as eczema and psoriasis, infectious
diseases like fungal infections and bacterial dermatoses, pigmentary
disorders such as vitiligo, and benign growths including seborrheic
keratoses and skin tags (\citeproc{ref-hay2006skin}{Hay et al., 2006}).
While these lesions are typically non-fatal, they can cause chronic
discomfort, psychological distress, and social stigma, especially when
affecting visible body areas.

In Africa, the high prevalence of infectious dermatoses is closely
linked to environmental and socioeconomic conditions. Fungal infections
such as tinea capitis are particularly common among children in rural
and peri-urban areas due to overcrowding, limited hygiene facilities,
and humid climatic conditions (\citeproc{ref-van2005common}{Hees \&
Naafs, 2005}). Similarly, bacterial skin infections like impetigo and
cellulitis are widespread, often arising as secondary infections in
individuals with compromised skin barriers from insect bites, eczema, or
other dermatological conditions.

Chronic inflammatory diseases such as eczema and psoriasis, though less
prevalent than infectious dermatoses, represent a significant health
burden due to their recurrent nature and the need for long-term
management. In many low-resource settings, limited access to
dermatologists and diagnostic tools leads to misdiagnosis, inappropriate
treatment, and poor disease control. Pigmentary disorders such as
vitiligo and post-inflammatory hyperpigmentation are also of particular
concern in African populations, where cultural perceptions and stigma
may influence healthcare-seeking behavior
(\citeproc{ref-hay2006skin}{Hay et al., 2006}).

From a diagnostic perspective, the visual similarity between
non-cancerous and cancerous lesions can pose a challenge, particularly
for general practitioners without specialized dermatological training.
AI-driven image analysis tools have the potential to assist in
differentiating these lesions, improving diagnostic accuracy in settings
where specialist input is scarce.

\subsubsection{Cancerous Lesions}\label{cancerous-lesions}

Cancerous lesions, though less common than non-cancerous forms, are of
critical concern due to their potential for metastasis and mortality.
The most prevalent malignant skin cancers include melanoma, squamous
cell carcinoma (SCC), and basal cell carcinoma (BCC). While melanoma is
less frequent in darker skin tones, its prognosis is often poorer in
African populations due to late-stage diagnosis
(\citeproc{ref-zafar2023skin}{Zafar et al., 2023}). SCC and BCC, on the
other hand, may occur in sun-exposed or chronically damaged skin
regardless of skin tone.

Globally, skin cancers represent a significant proportion of all cancer
diagnoses, with an estimated 1.5 million new cases annually
(\citeproc{ref-kassem2021machine}{Kassem et al., 2021}). In high-income
countries, public health campaigns and widespread access to
dermatological care have improved early detection rates, leading to
better survival outcomes. In contrast, in many African countries,
including Cameroon, public awareness of skin cancer remains low, and
access to diagnostic biopsies and histopathology services is limited.
Consequently, patients often present with advanced disease stages,
reducing treatment options and survival prospects.

Risk factors for skin cancer in African populations include albinism,
chronic scarring from burns or ulcers, prolonged exposure to UV
radiation (especially among outdoor workers), and certain viral
infections such as human papillomavirus (HPV). Individuals with
albinism, in particular, face a dramatically elevated risk of developing
SCC due to the absence of protective melanin in the skin.

Diagnosing skin cancer in low-resource settings is challenging due to
both infrastructural and human resource constraints. Dermoscopy,
histopathology, and other confirmatory diagnostic modalities may be
unavailable outside urban tertiary hospitals. In such contexts, mobile
AI-assisted diagnostic tools can play an instrumental role in triaging
suspicious lesions, guiding referrals, and facilitating earlier
intervention.

\subsection{Role of Artificial Intelligence in
Dermatology}\label{role-of-artificial-intelligence-in-dermatology}

Artificial Intelligence (AI) has emerged as a transformative force in
medical diagnostics, with dermatology standing out as a field
particularly well-suited to AI-driven interventions. This suitability
stems from the inherently visual nature of dermatological assessment,
where clinical diagnosis often relies heavily on visual inspection of
lesion morphology, color, texture, and distribution patterns. AI
systems, especially those based on computer vision, can replicate and,
in some cases, surpass human pattern recognition capabilities by
analyzing large volumes of annotated images
(\citeproc{ref-kassem2021machine}{Kassem et al., 2021}),
(\citeproc{ref-ogudo2023optimal}{Ogudo et al., 2023}).

In recent years, deep learning algorithms have demonstrated diagnostic
performance comparable to, and occasionally exceeding, that of
board-certified dermatologists
(\citeproc{ref-esteva2017dermatologist}{Esteva et al., 2017}). These
advancements have been facilitated by the availability of large-scale
annotated image datasets, improvements in computational power, and
algorithmic innovations in neural network architectures. AI-powered
systems have been successfully developed to identify and differentiate
between multiple skin lesion types, including both malignant and
non-malignant conditions, with high sensitivity and specificity.

Beyond diagnostic accuracy, AI offers significant advantages in
scalability and accessibility. Once trained, AI models can be deployed
on a wide range of devices from high-performance clinical workstations
to mobile smartphones, making them particularly valuable in low-resource
environments. In rural areas of Cameroon, for example, where access to
dermatologists is limited, AI-powered mobile applications could enable
preliminary screening and triage, directing patients with suspicious
lesions to specialized care.

Another important role of AI in dermatology lies in \textbf{decision
support}. AI systems can assist clinicians by highlighting regions of
interest, providing probability scores for different lesion classes, and
integrating clinical metadata (such as patient age, lesion history, and
risk factors) into predictive models. This reduces the likelihood of
oversight in busy clinical environments and supports more consistent
diagnostic outcomes.

Furthermore, AI-based lesion analysis is not limited to classification.
Emerging research is exploring its application in lesion segmentation,
disease progression tracking, and even predictive analytics for
treatment response. Such capabilities could support not only individual
patient management but also large-scale epidemiological surveillance.

Despite these strengths, AI adoption in dermatology is not without
challenges. Issues such as dataset bias, lack of diverse representation
(especially for darker skin tones), and limited interpretability of deep
learning models can affect trust and clinical uptake
(\citeproc{ref-gouda2022detection}{Gouda et al., 2022}). Ethical and
regulatory considerations ranging from patient data privacy to liability
in the event of misdiagnosis, also remain critical hurdles. Addressing
these concerns will be essential for ensuring that AI complements rather
than replaces clinical judgment, particularly in sensitive medical
contexts.

\subsection{Convolutional Neural Networks
(CNNs)}\label{convolutional-neural-networks-cnns}

Convolutional Neural Networks (CNNs) represent the cornerstone of modern
computer vision and have become the dominant architecture for image
classification tasks, including dermatological image analysis. Inspired
by the organization of the animal visual cortex, CNNs are designed to
automatically learn hierarchical feature representations directly from
raw pixel data, eliminating the need for manual feature engineering
(\citeproc{ref-lecun2015deep}{LeCun et al., 2015}).

A typical CNN architecture comprises several key components:

\begin{enumerate}
\def\labelenumi{\arabic{enumi}.}
\item
  \textbf{Convolutional Layers} -- These layers apply a series of
  learnable filters to the input image, producing feature maps that
  capture spatial hierarchies such as edges, textures, and shapes. Early
  layers detect simple patterns, while deeper layers learn more complex
  and abstract features relevant to classification.
\item
  \textbf{Pooling Layers} -- Pooling operations (e.g., max pooling,
  average pooling) reduce the spatial dimensions of feature maps,
  thereby decreasing computational load and controlling overfitting
  while retaining the most salient information.
\item
  \textbf{Activation Functions} -- Non-linear activation functions, such
  as ReLU (Rectified Linear Unit), introduce non-linearity into the
  network, enabling it to learn complex decision boundaries.
\item
  \textbf{Fully Connected Layers} -- These layers interpret the
  high-level features extracted by convolutional and pooling layers,
  ultimately producing class probability scores through a softmax or
  sigmoid activation function.
\end{enumerate}

The strength of CNNs in dermatology lies in their ability to extract
discriminative features from lesion images, even in the presence of
significant variability in lesion size, shape, color, and background
noise. This is particularly important for differentiating between
lesions with subtle morphological differences, such as differentiating
an atypical mole from early-stage melanoma.

In the context of skin lesion classification, CNNs have been
successfully applied to both binary classification tasks (e.g., benign
vs malignant) and multi-class classification covering a range of lesion
types (\citeproc{ref-kassem2021machine}{Kassem et al., 2021}). Their
performance is further enhanced through techniques such as \textbf{data
augmentation} (rotations, flips, scaling, color jittering) and
\textbf{regularization} (dropout, weight decay), which improve
generalization and robustness.

However, training CNNs from scratch requires large, balanced datasets,
an obstacle in medical imaging where data is often scarce and unevenly
distributed across classes. This limitation has driven widespread
adoption of \textbf{transfer learning}, wherein CNN architectures
pre-trained on large general-purpose datasets (such as ImageNet) are
fine-tuned on specific medical datasets. Models like ResNet, Inception,
and EfficientNet have been extensively leveraged in this manner,
yielding strong results even with limited medical image data.

For this project, CNNs form the foundational architecture of the lesion
detection and classification system, enabling automated feature
extraction and classification in a manner that is both scalable and
adaptable to diverse deployment environments. Their proven success in
similar tasks makes them a natural choice for addressing the diagnostic
challenges associated with skin lesions in Cameroon and other
low-resource settings.

\subsection{Transfer Learning and ResNet
Models}\label{transfer-learning-and-resnet-models}

Training deep learning models from scratch typically requires vast
datasets containing millions of annotated images. In medical imaging,
however, such large-scale datasets are rare due to privacy concerns, the
cost of expert annotations, and the limited prevalence of certain
conditions (\citeproc{ref-gouda2022detection}{Gouda et al., 2022}). This
scarcity makes \textbf{transfer learning} a practical and effective
approach for developing high-performing models with limited
domain-specific data.

Transfer learning involves leveraging a model pre-trained on a large,
general-purpose dataset such as ImageNet, which contains over 14 million
images across 1,000 classes and adapting it to a new but related task
(\citeproc{ref-lecun2015deep}{LeCun et al., 2015}). In this paradigm,
the lower layers of the pre-trained model, which capture generic
features like edges, textures, and shapes, are retained, while the
higher layers, which learn task-specific features, are fine-tuned on the
target medical dataset. This approach reduces training time, lowers
computational requirements, and mitigates the risk of overfitting.

Among the various architectures used for transfer learning,
\textbf{Residual Networks (ResNet)} have become particularly prominent
due to their ability to train very deep networks without succumbing to
the vanishing gradient problem (\citeproc{ref-he2016deep}{He et al.,
2016}). ResNet's innovation lies in its \emph{residual blocks}, which
introduce shortcut connections that allow gradients to flow more easily
through the network during backpropagation. This architecture enables
the training of models with hundreds of layers while maintaining
stability and accuracy.

ResNet-18, ResNet-34, and ResNet-50 are among the most widely used
variants in medical imaging applications, each offering a trade-off
between computational complexity and representational power. For skin
lesion classification, ResNet architectures have consistently
demonstrated strong performance, particularly when fine-tuned with
domain-specific data and augmented through preprocessing techniques such
as rotation, scaling, and color normalization
(\citeproc{ref-kassem2021machine}{Kassem et al., 2021}).

In the context of this study, ResNet-18 was selected for its balance
between accuracy and computational efficiency, making it suitable for
deployment on both research environments and resource-constrained
devices like smartphones. Transfer learning with ResNet-18 allows the
model to benefit from robust, pre-learned visual features while adapting
to the nuances of dermatological imagery from diverse sources, including
those representing darker skin tones and non-cancerous lesions that are
often underrepresented in global datasets.

\subsection{Dataset Challenges in Medical
Imaging}\label{dataset-challenges-in-medical-imaging}

The success of AI systems in medical imaging is intrinsically tied to
the quality, diversity, and representativeness of the datasets used for
training and evaluation. In dermatology, this presents several
significant challenges.

\textbf{1. Data Scarcity and Class Imbalance} -- High-quality, annotated
dermatological datasets are limited, particularly for rare conditions
and underrepresented demographics. Publicly available datasets such as
those on Kaggle or DermNetNZ often contain disproportionate numbers of
images for certain lesion types, leading to class imbalance. Models
trained on such datasets risk becoming biased toward majority classes,
resulting in reduced accuracy for minority categories
(\citeproc{ref-gouda2022detection}{Gouda et al., 2022}).

\textbf{2. Limited Skin Tone Representation} -- Many benchmark datasets
are heavily skewed toward lighter skin tones, reflecting their origins
in high-income, predominantly Caucasian populations
(\citeproc{ref-gouda2022detection}{Gouda et al., 2022}). This lack of
diversity can lead to systematic biases, where models perform well on
lighter skin but poorly on darker tones---a critical limitation when
deploying AI tools in African contexts.

\textbf{3. Variability in Image Acquisition} -- Images in dermatology
can be captured under a wide range of conditions, including variations
in lighting, background, focus, and resolution. Differences in equipment
from professional dermatoscopes to smartphone cameras, introduce further
heterogeneity, making model generalization more difficult
(\citeproc{ref-zafar2023skin}{Zafar et al., 2023}). Robust preprocessing
pipelines, including color correction, normalization, and augmentation,
are essential to mitigate these effects.

\textbf{4. Annotation Quality and Consistency} -- Accurate labeling of
lesion types often requires expert dermatological input. In public
datasets, annotations may be inconsistent or based solely on visual
inspection rather than biopsy-confirmed diagnoses. This introduces noise
into the training data, potentially lowering model performance
(\citeproc{ref-van2005common}{Hees \& Naafs, 2005}).

\textbf{5. Ethical and Privacy Concerns} -- Medical image datasets must
comply with strict privacy regulations to protect patient identities.
This can limit the sharing of comprehensive datasets, particularly those
including metadata such as patient age, sex, and medical history.
De-identification processes, while necessary, may also remove
potentially valuable contextual information.

For this project, these challenges were addressed through
\textbf{multi-source dataset compilation}, selecting high-quality images
from multiple Kaggle repositories and the DermNetNZ image database.
Images were chosen based on clarity, annotation reliability, and
diversity in lesion presentation. Preprocessing steps, including
augmentation techniques, were applied to improve model robustness. While
this approach does not entirely eliminate dataset-related limitations,
it provides a strong foundation for training a model capable of handling
real-world variability in skin lesion imagery.

\subsubsection{Review of Existing
Systems}\label{review-of-existing-systems}

The development of automated skin lesion detection systems has evolved
significantly over the past three decades, transitioning from rule-based
image processing algorithms to advanced deep learning frameworks capable
of multi-class classification. Early computer-aided diagnosis (CAD)
systems, developed in the late 1990s and early 2000s, primarily relied
on handcrafted features such as color histograms, shape descriptors, and
texture measures, combined with classical classifiers like Support
Vector Machines (SVMs) and k-Nearest Neighbors (k-NN). While these
systems achieved moderate success in controlled environments, their
reliance on manually engineered features made them highly sensitive to
variations in lighting, resolution, and lesion morphology
(\citeproc{ref-gouda2022detection}{Gouda et al., 2022}).

Over the past decade, deep learning, particularly Convolutional Neural
Networks (CNNs) has revolutionized dermatological image analysis.
CNN-based models eliminate the need for manual feature extraction by
learning hierarchical feature representations directly from pixel data
(\citeproc{ref-lecun2015deep}{LeCun et al., 2015}). Notable milestones
include the work of Esteva et al., who demonstrated dermatologist-level
classification of skin cancer using a CNN trained on over 129,000
clinical images (\citeproc{ref-esteva2017dermatologist}{Esteva et al.,
2017}), and the introduction of transfer learning approaches that
significantly reduce the data requirements for effective model training
(\citeproc{ref-he2016deep}{He et al., 2016}).

Several prominent publicly available systems illustrate the current
state of the art. For example, the \textbf{ISIC (International Skin
Imaging Collaboration)} challenge platforms have driven substantial
progress by providing standardized datasets and benchmarking
opportunities for skin lesion classification and segmentation tasks.
Solutions from top-performing teams often integrate deep CNN
architectures such as ResNet, Inception, and EfficientNet, combined with
ensemble learning and advanced data augmentation techniques.

In commercial and practical deployment contexts, mobile applications
such as \emph{SkinVision} and \emph{Miiskin} leverage AI to provide
lesion risk assessments directly to users via smartphone cameras. While
these tools have expanded public access to preliminary screening, they
often focus on binary classification (e.g., suspicious
vs.~non-suspicious) and may not comprehensively cover the range of
benign and malignant lesion types relevant in diverse populations,
particularly in low-resource settings like Cameroon.

Within academic research, several studies have explored multi-class
lesion classification. Kassem et al.~proposed a CNN-based framework
achieving high accuracy across seven lesion classes in the HAM10000
dataset (\citeproc{ref-kassem2021machine}{Kassem et al., 2021}), while
Ogudo et al.~adapted a transfer learning approach for African skin
tones, underscoring the importance of dataset diversity
(\citeproc{ref-ogudo2023optimal}{Ogudo et al., 2023}). These studies
highlight the growing focus on inclusivity and generalizability in AI
dermatology systems.

Despite these advancements, limitations persist. Many existing systems
are trained on datasets that lack representation of darker skin tones,
have a narrow lesion type coverage, or fail to address deployment
constraints such as offline functionality and low computational
resources. Furthermore, while commercial solutions often reach end-users
quickly, they may not undergo the same level of clinical validation or
peer-reviewed scrutiny as academic research outputs.

\subsection{Research Gap}\label{research-gap}

While the progress in AI-powered dermatological diagnostics over the
last decade has been remarkable, several gaps remain, particularly in
the context of African healthcare systems.

\textbf{1. Dataset Representation} -- Most high-performing models are
trained on datasets dominated by images from lighter skin tones. This
limits their diagnostic accuracy when applied to African populations,
where lesion appearance can differ significantly due to higher melanin
content and unique environmental exposure patterns
(\citeproc{ref-gouda2022detection}{Gouda et al., 2022}). The lack of
comprehensive datasets covering both malignant and non-malignant lesions
in darker skin is a critical barrier.

\textbf{2. Multi-Class Coverage} -- Many commercial and academic systems
prioritize melanoma detection due to its high mortality rate in
fair-skinned populations. However, in African contexts, non-cancerous
lesions, such as fungal infections, eczema, and pigmentary disorders,
constitute the majority of dermatological cases and therefore require
equal diagnostic attention.

\textbf{3. Deployment in Low-Resource Settings} -- Existing systems
often assume reliable internet connectivity and high-performance
computing infrastructure, making them unsuitable for rural or
underserved regions. Offline-capable, computationally efficient models
are needed to bridge this gap.

\textbf{4. Clinical Integration and Trust} -- Limited collaboration
between AI developers and healthcare providers in Africa has slowed the
integration of AI tools into clinical workflows. Moreover, clinicians
may be hesitant to adopt AI systems without clear interpretability,
validation on local datasets, and regulatory approval.

\textbf{5. Comprehensive Diagnostic Workflow} -- While classification
accuracy is a major focus of existing research, other aspects of the
diagnostic process such as lesion segmentation, tracking over time, and
integration with patient health records, remain underexplored in
resource-constrained contexts.

This study addresses these gaps by developing a \textbf{multi-class skin
lesion classification system} optimized for deployment in Cameroon. The
approach combines a curated dataset from multiple open-access sources
(Kaggle repositories and DermNetNZ), robust preprocessing and
augmentation, transfer learning using ResNet-18, and deployment pathways
that accommodate both research and mobile application environments. By
doing so, it aims to improve diagnostic accuracy across a broader range
of lesions, enhance representation for darker skin tones, and enable
accessibility in low-resource healthcare settings.

\subsection{Summary}\label{summary}

This chapter has provided a comprehensive review of the literature
relevant to the development of AI-powered skin lesion detection and
classification systems. The discussion began with an overview of skin
lesions, distinguishing between non-cancerous and cancerous categories,
and emphasizing their prevalence and diagnostic challenges both globally
and within the African context. The role of AI in dermatology was
examined, with a particular focus on CNN architectures, transfer
learning strategies, and the ResNet model family, which has proven
highly effective in medical imaging tasks.

Dataset challenges in medical imaging were explored in detail,
highlighting issues such as class imbalance, limited representation of
darker skin tones, variability in image acquisition, and annotation
quality. The review of existing systems illustrated the evolution of
lesion detection technologies from early handcrafted-feature approaches
to modern deep learning-based solutions, while also identifying their
limitations in terms of inclusivity, deployment feasibility, and
comprehensive lesion coverage.

Finally, the research gap analysis underscored the need for multi-class,
skin-tone-inclusive, resource-efficient diagnostic systems that can be
deployed effectively in low-resource settings such as Cameroon. The
following chapter will present the methodology adopted for this study,
detailing the dataset compilation, preprocessing, model training,
evaluation, and deployment strategies used to develop the proposed skin
lesion classification system.

\newpage

\section{Chapter 3: Methodology}\label{chapter-3-methodology}

\begin{itemize}
\tightlist
\item
  3.1 Introduction
\item
  3.2 Project Framework and Approach (e.g., CRISP-DM or Nunamaker)
\item
  3.3 Dataset Description and Preprocessing
\item
  3.4 Model Architecture (ResNet18 and Modifications)
\item
  3.5 Training Strategy
\item
  3.6 Evaluation Metrics
\item
  3.7 Tools and Libraries Used
\item
  3.8 Summary
\end{itemize}

\newpage

\section{Chapter 4: System Design and
Implementation}\label{chapter-4-system-design-and-implementation}

\begin{itemize}
\tightlist
\item
  4.1 System Overview
\item
  4.2 Functional Requirements
\item
  4.3 System Architecture Diagram
\item
  4.4 Data Flow and Pipeline
\item
  4.5 CLI and GUI Components
\item
  4.6 Integration and Deployment
\item
  4.7 Inference and Prediction Logic
\item
  4.8 Summary
\end{itemize}

\newpage

\section{Chapter 5: Testing, Results, and
Discussion}\label{chapter-5-testing-results-and-discussion}

\begin{itemize}
\tightlist
\item
  5.1 Dataset Splits (Train, Validation, Test)
\item
  5.2 Accuracy, Loss, and Confusion Matrix
\item
  5.3 ROC, Precision, Recall, F1-score
\item
  5.4 Model Limitations and Observed Bias
\item
  5.5 Comparison with Existing Systems
\item
  5.6 Summary
\end{itemize}

\newpage

\section{Chapter 6: Conclusion and
Recommendations}\label{chapter-6-conclusion-and-recommendations}

\begin{itemize}
\tightlist
\item
  6.1 Summary of Findings
\item
  6.2 Conclusion
\item
  6.3 Contributions of the Study
\item
  6.4 Recommendations for Future Work
\item
  6.5 Final Remarks
\end{itemize}

\newpage

\section{Appendices}\label{appendices}

\begin{itemize}
\tightlist
\item
  Appendix A: Source Code Snippets
\item
  Appendix B: Model Configurations
\item
  Appendix C: Training Graphs
\item
  Appendix D: Screenshots of GUI
\item
  Appendix E: User Manual
\end{itemize}

\newpage

\section*{References}\label{references}
\addcontentsline{toc}{section}{References}

\phantomsection\label{refs}
\begin{CSLReferences}{1}{0}
\bibitem[\citeproctext]{ref-adegun2020fcn}
Adegun, A. A., \& Viriri, S. (2020). FCN-based DenseNet framework for
automated detection and classification of skin lesions in dermoscopy
images. \emph{IEEE Access}, \emph{8}, 150377--150396.

\bibitem[\citeproctext]{ref-albahar2019skin}
Albahar, M. A. (2019). Skin lesion classification using convolutional
neural network with novel regularizer. \emph{IEEE Access}, \emph{7},
38306--38313.

\bibitem[\citeproctext]{ref-esteva2017dermatologist}
Esteva, A., Kuprel, B., Novoa, R. A., Ko, J., Swetter, S. M., Blau, H.
M., \& Thrun, S. (2017). Dermatologist-level classification of skin
cancer with deep neural networks. \emph{Nature}, \emph{542}(7639),
115--118.

\bibitem[\citeproctext]{ref-gouda2022detection}
Gouda, W., Sama, N. U., Al-Waakid, G., Humayun, M., \& Jhanjhi, N. Z.
(2022). Detection of skin cancer based on skin lesion images using deep
learning. \emph{Healthcare}, \emph{10}, 1183.

\bibitem[\citeproctext]{ref-harangi2018skin}
Harangi, B. (2018). Skin lesion classification with ensembles of deep
convolutional neural networks. \emph{Journal of Biomedical Informatics},
\emph{86}, 25--32.

\bibitem[\citeproctext]{ref-hay2006skin}
Hay, R., Bendeck, S. E., Chen, S., Estrada, R., Haddix, A., McLeod, T.,
\& Mahé, A. (2006). Skin diseases. \emph{Disease Control Priorities in
Developing Countries. 2nd Edition}.

\bibitem[\citeproctext]{ref-he2016deep}
He, K., Zhang, X., Ren, S., \& Sun, J. (2016). Deep residual learning
for image recognition. \emph{Proceedings of the IEEE Conference on
Computer Vision and Pattern Recognition}, 770--778.

\bibitem[\citeproctext]{ref-van2005common}
Hees, C. van, \& Naafs, B. (2005). \emph{Common skin diseases africa}.

\bibitem[\citeproctext]{ref-jinnai2020development}
Jinnai, S., Yamazaki, N., Hirano, Y., Sugawara, Y., Ohe, Y., \&
Hamamoto, R. (2020). The development of a skin cancer classification
system for pigmented skin lesions using deep learning.
\emph{Biomolecules}, \emph{10}(8), 1123.

\bibitem[\citeproctext]{ref-kassem2021machine}
Kassem, M. A., Hosny, K. M., Damaševičius, R., \& Eltoukhy, M. M.
(2021). Machine learning and deep learning methods for skin lesion
classification and diagnosis: A systematic review. \emph{Diagnostics},
\emph{11}(8), 1390.

\bibitem[\citeproctext]{ref-khan2021multi}
Khan, M. A., Muhammad, K., Sharif, M., Akram, T., \& Albuquerque, V. H.
C. de. (2021). Multi-class skin lesion detection and classification via
teledermatology. \emph{IEEE Journal of Biomedical and Health
Informatics}, \emph{25}(12), 4267--4275.

\bibitem[\citeproctext]{ref-lecun2015deep}
LeCun, Y., Bengio, Y., \& Hinton, G. (2015). Deep learning.
\emph{Nature}, \emph{521}(7553), 436--444.

\bibitem[\citeproctext]{ref-mahbod2019skin}
Mahbod, A., Schaefer, G., Wang, C., Ecker, R., \& Ellinge, I. (2019).
Skin lesion classification using hybrid deep neural networks.
\emph{ICASSP 2019-2019 IEEE International Conference on Acoustics,
Speech and Signal Processing (ICASSP)}, 1229--1233.

\bibitem[\citeproctext]{ref-ogudo2023optimal}
Ogudo, K. A., Surendran, R., \& Khalaf, O. I. (2023). Optimal artificial
intelligence based automated skin lesion detection and classification
model. \emph{Computer Systems Science \& Engineering}, \emph{44}(1).

\bibitem[\citeproctext]{ref-yap2018multimodal}
Yap, J., Yolland, W., \& Tschandl, P. (2018). Multimodal skin lesion
classification using deep learning. \emph{Experimental Dermatology},
\emph{27}(11), 1261--1267.

\bibitem[\citeproctext]{ref-zafar2023skin}
Zafar, M., Sharif, M. I., Sharif, M. I., Kadry, S., Bukhari, S. A. C.,
\& Rauf, H. T. (2023). Skin lesion analysis and cancer detection based
on machine/deep learning techniques: A comprehensive survey.
\emph{Life}, \emph{13}(1), 146.

\end{CSLReferences}

\end{document}
