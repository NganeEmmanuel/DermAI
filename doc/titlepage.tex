
\begin{titlepage}
    \centering
    
    % University logo (replace with actual path)
    \IfFileExists{ictu-logo.png}{
        \includegraphics[width=0.5\textwidth]{ictu-logo.png}
    }{
        \vspace*{1cm}
        \textbf{\Large THE ICT UNIVERSITY}
    }
    
    \vspace{1cm}
    
    {\Large Faculty of Information and Communication Technology}
    
    \vspace{1.2cm}
    
    {A dissertation presented and submitted in partial fulfilment of the requirements\\
    for the degree of a Bachelor of Science in Computer Science}
    
    \vspace{0.5cm}

    {Titile}
    
    \textbf{Development of a Skin Lesion Detection and Classification System using Convolutional Neural Networks (CNNs)}
    
    \vspace{0.4cm}
    
    \textbf{By}
    
    \text{Ngane Emmanuel}
    
    \vspace{0.5cm}
    
    {Registration Number: ICTU20222972}
    
    \vspace{1cm}
    
    {Supervised by: Engr. Nkiamboh Tanwi}
    
    \vspace{2cm}
    
    \textbf{\today}
    
    \vfill
\end{titlepage}
\clearpage
\thispagestyle{empty}
\begin{center}
\textbf{\large DECLARATION}
\end{center}

\vspace{1cm}

I declare that the work entitled \textbf{``Development of a Skin Lesion Detection and Classification System using Convolutional Neural Networks (CNNs)''} is my own original work, conceived and presented in the partial fulfilment of the requirement for the degree of a Bachelor of Science in Computer Science at ICT University. This work has not been submitted for any degree or examination in any other university, and that all the sources I have used or quoted have been indicated and acknowledged as complete references.

\vspace{1.2cm}

\begin{tabular}{ll}
Signed \rule{4cm}{0.15mm} \hspace{3cm} & Date: \rule{4cm}{0.15mm} \\ [0.5cm]
Name: \rule{4cm}{0.15mm} & \\ [0.5cm]
Registration Number: \rule{4cm}{0.15mm} & \\
\end{tabular}
\vfill
\clearpage
\thispagestyle{empty}
\begin{center}
\textbf{\large CERTIFICATION}
\end{center}

\vspace{1cm}

This work entitled \textbf{``Development of a Skin Lesion Detection and Classification System using Convolutional Neural Networks (CNNs)''} has been submitted for examination with my approval as the Research Supervisor.

\vspace{2cm}

\begin{tabular}{ll}
Signed \rule{4cm}{0.15mm} \hspace{3cm} & Date: \rule{4cm}{0.15mm} \\ [0.5cm]
Name: \rule{4cm}{0.15mm} & \\
\end{tabular}
\clearpage

\thispagestyle{empty}
\begin{center}
\textbf{\large DEDICATION}
\end{center}

\vspace{2cm} % Adjust spacing as needed

I dedicate this work to my parents, Rev. Dr. Ntoko Samuel Eseh and Mme Ntoko Grace Melioge, and to my loving sister, Ntoko Racheal Edenge, for their unwavering support, encouragement, and sacrifices throughout my academic journey.

\vfill % Pushes the text to the vertical center
\clearpage

\thispagestyle{empty}
\begin{center}
\textbf{\large Acknowledgments}
\end{center}

\vspace{2cm} % Adjust spacing as needed

I would like to express my sincere appreciation to my project supervisor, Engr. Nkiambo Tanwi, whose guidance, feedback, and encouragement were invaluable throughout the course of this project. His expertise and support helped shape both the direction and quality of this research.

I am also grateful to the faculty and staff of the Department of Computer Science, ICT University, for providing the academic foundation and resources necessary for the completion of this work. Their commitment to academic excellence has been instrumental in my development.

Special thanks go to my classmates and friends for their collaboration, feedback, and moral support during the challenging phases of this project. Their insights and motivation helped me stay focused and persistent.

Lastly, I appreciate my family for their continuous encouragement and understanding, which enabled me to dedicate the necessary time and energy to this project.

This project has been a significant learning experience, and I am thankful to all who contributed to its successful completion.

\vfill % Pushes the text to the vertical center
\clearpage

\thispagestyle{empty}
\begin{center}
\textbf{\large FACULTY APPROVAL}
\end{center}

\vspace{1cm}

This dissertation has been duly reviewed by the Department and the Faculty and is ready for examination with our approval.

\vspace{2cm}

\begin{flushright}
\textbf{Approved by}

\vspace{1.5cm}

\begin{tabular}{@{}r@{}}
Signature \hspace{3cm} Date \\ [0.5cm]
\rule{6cm}{0.15mm} \\ [0.2cm]
Engr. Nkiamboh Tanwi \\ [0.3cm]
Supervisor \\[1cm]

Signature \hspace{3cm} Date \\ [0.5cm]
\rule{6cm}{0.15mm} \\ [0.2cm]
Dr. Abdallah Ziraba \\ [0.3cm]
Head of Department \\[1cm]

Signature \hspace{3cm} Date \\ [0.5cm]
\rule{6cm}{0.15mm} \\ [0.2cm]
Dr. Luc Einstein Ngend \\ [0.3cm]
Dean \\
\end{tabular}
\end{flushright}

\vfill
\clearpage

% Abstract Page
\thispagestyle{empty}
\begin{center}
\textbf{\Large ABSTRACT}
\end{center}

\vspace{1cm}

Skin lesions encompass a wide spectrum of conditions, ranging from benign disorders such as eczema, fungal infections, and psoriasis to malignant forms like melanoma and squamous cell carcinoma. Globally, these conditions impose significant public health and economic burdens, with their impact being particularly acute in low-resource regions where specialized dermatological expertise is scarce. In Cameroon and much of sub-Saharan Africa, access to timely and accurate dermatological diagnosis is hindered by limited specialist availability, under-resourced health facilities, and geographic barriers to care. 

This study presents the development of a multi-class skin lesion detection and classification system using Convolutional Neural Networks (CNNs), designed to enhance diagnostic accessibility through automated image analysis. The system was trained on a curated, high-quality dataset compiled from multiple open-access sources, including Kaggle dermatology repositories and the DermNetNZ medical image database. Careful preprocessing and augmentation techniques were employed to normalize image quality, address class imbalance, and improve model robustness across diverse lesion presentations and skin tones.

A fine-tuned ResNet-18 architecture, leveraging transfer learning from the ImageNet dataset, was implemented to classify lesions into multiple diagnostic categories, covering both malignant and non-malignant conditions. The model was trained using supervised learning with categorical cross-entropy loss, and evaluated using metrics including accuracy, precision, recall, and F1-score. The final system achieved a validation accuracy of 75\% and demonstrated consistent performance on the test set, underscoring its potential as a supportive diagnostic aid.

The implementation includes a command-line interface (CLI) for research use and a prototype mobile application framework aimed at offline deployment in rural and underserved settings. By combining computational efficiency with diagnostic breadth, the system offers a cost-effective and scalable approach to skin lesion classification in contexts where traditional healthcare access is limited. 

This project contributes to the growing field of AI-assisted dermatology by emphasizing inclusivity across skin tones and lesion types, while maintaining technical feasibility for low-resource environments. Future improvements will focus on expanding the dataset, integrating explainable AI features, incorporating multi-modal patient data, and validating the system in real-world clinical settings.

\vspace{1cm}
\noindent\textbf{Keywords:} Skin lesion detection, convolutional neural networks, ResNet-18, transfer learning, dataset diversity, mobile health, Cameroon, medical imaging.

\vfill
\clearpage
